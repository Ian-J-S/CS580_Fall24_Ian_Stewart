\documentclass{article}
\title{CSCI-580 Homework 2}
\author{Ian Stewart}
\date{October 8, 2024}

\usepackage[a4paper, total={6.2in, 8in}]{geometry}

% For pseudocode
\usepackage[noend]{algpseudocode}
\renewcommand\algorithmicthen{}
\renewcommand\algorithmicdo{}

% For embedding links
\usepackage[colorlinks, urlcolor=blue]{hyperref}

% For including graphics/pictures in doc
\usepackage{graphicx}
\graphicspath{./Figures}

\begin{document}

\maketitle

\section*{Part 1}

\begin{enumerate}
    \item Approximating $\pi$ using Monte Carlo methods.
    \begin{enumerate}
        \item Given the area of a circle $A_c = \pi r^2$ and a square
        with a side length of $r$ and an area of $A_s = (2r)^2$, we can
        take the ratio of the area of the circle to the area of the square
        is $\frac{\pi r^2}{4r^2} = \frac{\pi}{4}$. Using the ratio
        $4\frac{A_c}{A_s} = \pi$, we can take advantage of Monte Carlo
        methods to compute an approximation of $\pi$. The process is to
        calculate a large number of random points where
        $-1 \leq x \leq 1$ and $-1 \leq y \leq 1$. We then check each point
        to see if it is inside of the circle $x^2 + y^2 = 1$. To approximate
        $\pi$ we take advantage of the ratio of areas, but instead take the ratio
        of the number of points inside the circle to the total number of points,
        multiplied by 4.
        \item Here is a C++ implementation of a function to approximate $\pi$.
        \begin{flushleft}
            \includegraphics[scale=0.20]{Figures/approxPi3.png}
        \end{flushleft}
        \newpage
        \item Here are two figures showing the improved approximation as
        sample size increases. I have included one without error bars and
        one with error bars.
    \end{enumerate}
    \begin{center}
        \includegraphics[scale=0.85]{Figures/approx_pi.png}
        \includegraphics[scale=0.85]{Figures/approx_pi_error_bars.png}
    \end{center}

    \newpage
    \section*{Part 2}
    \item BFS and IDA* are similar in that both systematically and incrementally
    explore their search space. In BFS, all nodes at a given depth are explored
    before moving to the next level. Similarly, in IDA*, nodes are explored up to
    a specific depth threshold before the search deepens with a higher limit. This
    incremental exploration in both algorithms ensures that the search progresses
    in a controlled and organized manner, covering all possibilities within the
    current depth or depth threshold. \\
    The line between DFS and IDA* signifies the fact that IDA* will explore in a
    depth-first manner up to the depth threshold specified. It is similar to minimax
    in that minimax also uses a depth-first traversal order. Another similarity is
    that both use backtracking to travel back up the trees they are exploring. \\
    Dijkstra's algorithm is similar to alpha-beta pruning because they are both
    optimization algorithms that reduce unnecessary computations. Dijkstra's does
    this by always evaluating the shortest path first, and alpha-beta pruning does
    this by pruning tree branches that won't affect the minimax values. \\
    A* is similar to IDA* due to the fact that they both incorporate a heuristic
    to aid with path evaluation. More specifically, they both aim to encode some
    form of prior knowledge to increase the efficiency of the algorithms. \\
    D* has a similar dynamic nature to IDA*. IDA* changes its threshold dynamically
    as the search deepens, whereas D* can dynamically readjust its pathfinding
    in a changing environment. They also both use heuristic-driven approaches to aid
    in exploration. \\
    R* and Monte Carlo tree search have a line between them because they both use
    a randomized approach. They both utilize randomness for problems in which it is
    infeasible to implement a deterministic solution. The line between R* and
    Softmax signifies the probabilistic approach used in both approaches. R*
    can be viewed as probabilistic because of the randomness of its exploration.
    Softmax assigns probabilities to each option and can be adjusted via the
    "temperature" to control the likelihood of an exploitative option being
    chosen. \\
    I believe there could be another connecting line between A* (Heuristic driven)
    and Softmax search. This is because the temperature value of Softmax could be
    thought of as a sort of heuristic. By tuning the temperature, a person is
    essentially encoding their prior knowledge of the problem to influence the
    search process.
\end{enumerate}
\end{document}